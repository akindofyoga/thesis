\chapter{Conclusion and Future Work}\label{chap:conclusion}

This chapter concludes the dissertation by summarizing our contributions and
discussing open problems and future work on WCA applications.

\section{Contributions}

We address the problem of \textit{Scaling Up Wearable Cognitive Assistance for
  Assembly Tasks} that involve many parts.
The thesis we validated is:

\textbf{
  Scaling up WCA to complex assembly tasks is challenging because of
  (a) the difficulty of
  vision-based state detection with very small parts in the context of much
  larger objects being assembled; (b) the combinatorial explosion
  of possible error states; and (c) the large manual effort needed to create
  accurate DNNs that can reliably determine when task steps have been completed.
  These problems can be solved by a combination of (1) hierarchical
  decomposition of
  complex assemblies into modular compositions of subassemblies, (2) on-demand
  seamless
  escalation for live expert assistance, and (3) synthetic generation of
  training
  sets for born-digital components. The resulting solution can be implemented in
  a scalable and maintainable way using modular software components.
  This will enable the development of WCA applications for more complex tasks,
  which is a necessary step along the path towards making WCA applications
  practical for real world tasks.
}

We propose computer vision techniques that make it possible to detect when steps
of these long tasks have been completed.
Then we demonstrate the feasibility of training computer vision models for WCA
using synthetic images.
To handle the combinatorial explosion in the number of errors states for a task,
we support escalation to a human task expert, and we created tools to determine
the number of experts required to keep queuing times reasonable.
We then developed a software framework that can be used to create new WCA
applications.
These contributions mark an important step in making WCA applications practical.
Developers need a way to support real world assembly tasks, without spending an
unreasonable amount of time collecting an labeling training images.
WCA applications need some way to guide users who have made errors completing
tasks.
Our software framework enables the deployment of scalable WCA deployments.

\section{Future Work}

\subsection{Subassembly Identification}

\citet{subassembly_identification} automatically identified subassemblies of
assembled objects based on CAD files.
Their work focused on assembly sequence
planning, rather than detecting completed task steps using computer vision.
Subassembly Identification that focused on making it easy to detect completed
task steps using computer vision would be a useful tool for Wearable Cognitive
Assistance.
Techniques that did not require a CAD file for the assembly would also be
helpful.

\subsection{Detecting Environmental Issues}

Our computer vision models might output an incorrect label if the object being
assembled is positioned at an awkward angle, or if the lighting in a room is
too dim.
WCA applications have no way of detecting when one of these issues has
occurred.
However, if the applications could detect issues like this, they could alert the
user.
The user could then correct the issue, and avoid the incorrect computer vision
results that the issue would have caused.

\subsection{Computer Vision Techniques}

Deep Neural Networks perform best when the training, validation, and testing
data are drawn from the same distribution as the data that the trained model
will be used on~\cite{pmlr-v97-recht19a}.
Unfortunately, this will rarely be the case in practice for WCA applications.
The lighting conditions that the application is used in and defects in how parts
are manufactured can introduce biases in the data that the application will see
at runtime.
Biases that did not exist in the training data can cause the DNNs used by the
application to perform poorly.
Each task that a developer creates a WCA application for requires its own
training set.
This limits the size of the training set that would be practical to collect in
order to develop one of these applications.
Computer vision models that are robust to biases that did not exist in the
original training set will improve the reliability of WCA applications.

Models that can adapt to environmental conditions that were not represented in
training data, such as different lighting, are also critical for making WCA
applications practical.
The models used in this work require developers to collect training data in a
variety of different environments.
This is sufficient for our assumption that a user's environment is well lit.
However, WCA applications would also be useful in more difficult environments.

The computer vision models used by our applications are trained to identify the
completed step shown in an image.
But they are not trained on any data depicting incomplete steps.
Our applications currently avoid telling a user that a step has been completed,
while the user is actually in the middle of the step, using the techniques
described in \S\ref{sec:implementation_details}.
In addition, users users hands often cover up parts of an assembly while they
are completing steps of a task, which prevents the object detector from finding
the subassembly in an image.
However, additional work is needed to reliably prevent the application from
giving the next instruction while the user is still in the middle of the
previous step of the task.

\subsection{Textures for 3D Models}

CAD designs for born-digital objects specify the shapes of parts.
However, they rarely include information about the materials that objects are
manufactured from.
In fact, a single CAD design can be used to manufacture objects out of multiple
different materials.
A person who wants to generate synthetic data for a new born-digital object must
specify texture information for the material that the object was made out of.
This is a time consuming process.
In particular, the person must ensure that the object looks realistic in a
variety of simulated lighting conditions.
Reducing the manual effort required to specify texture information will make it
easier to generate synthetic training images.
